%% Exemplo de utilizacao do estilo de formatacao normas-utf-tex (http://normas-utf-tex.sourceforge.net)
%% Autores: Hugo Vieira Neto (hvieir@utfpr.edu.br)
%%          Diogo Rosa Kuiaski (diogo.kuiaski@gmail.com)
%% Colaboradores:
%%          Cézar M. Vargas Benitez <cesarvargasb@gmail.com>
%%          Marcos Talau <talau@users.sourceforge.net>


\documentclass[openright]{normas-utf-tex} %openright = o capitulo comeca sempre em paginas impares
%\documentclass[oneside]{normas-utf-tex} %oneside = para dissertacoes com numero de paginas menor que 100 (apenas frente da folha) 


\usepackage[alf,abnt-emphasize=bf,bibjustif,recuo=0cm, abnt-etal-cite=2, abnt-etal-list=99]{abntcite} %configuracao correta das referencias bibliograficas.

\usepackage[brazil]{babel} % pacote portugues brasileiro
\usepackage[utf8]{inputenc} % pacote para acentuacao direta
\usepackage{amsmath,amsfonts,amssymb} % pacote matematico
\usepackage{graphicx} % pacote grafico
\usepackage{times} % fonte times
\usepackage{listings} % para código-fonte
\usepackage{algorithmic}

% configuracao do pacote listings
\lstset{
	basicstyle=\footnotesize,
	showstringspaces=false,
	tabsize=4,
	numbers=left,
	numberstyle=\tiny,
	stepnumber=1,
	numbersep=5pt,
	breaklines=true,
	extendedchars=true,
	frame=tb
}

%Podem utilizar GEOMETRY{...} para realizar pequenos ajustes das margens. Onde, left=esquerda, right=direita, top=superior, bottom=inferior. P.ex.:
%\geometry{left=3.0cm,right=1.5cm,top=4cm,bottom=1cm} 

% ---------- Preambulo ----------
\instituicao{Universidade Tecnol\'ogica Federal do Paran\'a} % nome da instituicao
\departamento{Departamento Acadêmico de Informática}
\programa{Curso Superior de Engenharia de Computação} % nome do programa

\documento{Trabalho de Conclusão de Curso} % [Disserta\c{c}\~ao] ou [Tese]
\nivel{Graduação} % [Mestrado] ou [Doutorado]
\titulacao{Engenheiro} % [Mestre] ou [Doutor]

\titulo{\MakeUppercase{Planejador de Rotas com Sistema de Transporte Público}} % titulo do trabalho em portugues
\title{\MakeUppercase{Trip Planner with Public Transport System}} % titulo do trabalho em ingles

\autor{Bruno Weingraber} % autor do trabalho
\autordois{Lucas Campiolo Paiva}
\autortres{Luiz Gustavo Cardoso Ribeiro}
\cita{WEINGRABER, Bruno; PAIVA, Lucas C.; RIBEIRO, Luiz G. C.} % sobrenome (maiusculas), nome do autor do trabalho

\palavraschave{Teoria dos Grafos, Redes Multimodais, Menor Caminho, Transporte Público} % palavras-chave do trabalho
\keywords{Graph Theory, Multimodal Networks, Shortest Path, Public Transport} % palavras-chave do trabalho em ingles

\comentario{\UTFPRdocumentodata\ de graduação, apresentado à disciplina de Trabalho de Conclusão de Curso II, do \UTFPRprogramadata\ do \UTFPRdepartamentodata\ da \ABNTinstituicaodata\, como requisito parcial para obten\c{c}\~ao do grau de \UTFPRtitulacaodata.}



\orientador{Prof. PhD Murilo V. G. da Silva} % nome do orientador do trabalho
%\orientador[Orientadora:]{Nome da Orientadora} % <- no caso de orientadora, usar esta sintaxe
%\coorientador{Nome do Co-orientador} % nome do co-orientador do trabalho, caso exista
%\coorientador[Co-orientadora:]{Nome da Co-orientadora} % <- no caso de co-orientadora, usar esta sintaxe
%\coorientador[Co-orientadores:]{Nome do Co-orientador} % no caso de 2 co-orientadores, usar esta sintaxe
%\coorientadorb{Nome do Co-orientador 2}	% este comando inclui o nome do 2o co-orientador

\local{Curitiba} % cidade
\data{\the\year} % ano automatico


%---------- Inicio do Documento ----------
\begin{document}

\capa % geracao automatica da capa
\folhaderosto % geracao automatica da folha de rosto
%\termodeaprovacao % <- ainda a ser implementado corretamente

% agradecimentos (opcional)
\begin{agradecimentos}
Agradecemos ao nosso orientador, Murilo, pelos conselhos, ideias e motivação. Agradecemos também por lembrar da trilha sonora do Mário 64 e por nos ensinar que o melhor Zelda de todos é o do SNES.
\end{agradecimentos}

%resumo
\begin{resumo}
Este trabalho apresenta o desenvolvimento de uma ferramenta web para realizar planejamento de rotas através de transporte público. Ele contém uma discussão teórica de redes multimodais e descreve um algoritmo de menor caminho para elas. Apresenta as ferramentas utilizadas no desenvolvimento, do banco de dados de grafos até o sistema de controle de versão. Apresenta também a metodologia de trabalho da equipe e uma descrição do que foi desenvolvido e dos resultados obtidos.
\end{resumo}

%abstract
\begin{abstract}
This work presents the development of a web tool that aims to help the planning of public transport routes. It contains a theoretical discussion of multi-modal networks and describes an algorithm for finding shortest-paths on them. It presents the tools used on the development, from the graph database to the version control system. Also, it presents the team's work metodology and a description of the built software and the obtained results.
\end{abstract}

% listas (opcionais, mas recomenda-se a partir de 5 elementos)
\listadefiguras % geracao automatica da lista de figuras
\listadetabelas % geracao automatica da lista de tabelas
\listadesiglas % geracao automatica da lista de siglas

% sumario
\sumario % geracao automatica do sumario


%---------- Inicio do Texto ----------
% recomenda-se a escrita de cada capitulo em um arquivo texto separado (exemplo: intro.tex, fund.tex, exper.tex, concl.tex, etc.) e a posterior inclusao dos mesmos no mestre do documento utilizando o comando \input{}, da seguinte forma:

% aparentemente esse sloppy resolve o problema das margens e do texttt
% não tenho certeza sobre os efeitos colaterais dele, no entanto
% fiquemos atentos! :P
\sloppy

%---------- Primeiro Capitulo: Introdução ----------
\chapter{Introdução}
A situação do trânsito das médias e grandes cidades brasileiras vem se tornando cada vez mais grave.
Tempo e dinheiro são perdidos devido a imensos congestionamentos e, estes por sua vez, resultam em grandes gastos em combustível, manutenção veicular e rodoviária.

Atualmente existe um projeto de lei tramitando no Congresso Nacional, o PL nº 1.687 \cite{FlexaRibeiro2010}, que consiste principalmente em priorizar o transporte público em detrimento ao particular e também incentivar o transporte não-motorizado em detrimento do motorizado.
Em segundo plano, vincula o planejamento urbano ao sistema de transporte, fazendo com que as cidades cresçam com um sistema de transporte ordenado.

Segundo \cite{IPEA2011}, atualmente a cada 12 reais investidos no transporte particular, somente 1 é investido no transporte público.
Este projeto de lei busca inverter as prioridades.
Para ilustrar a cituação atual do sistema de transporte público brasileiro, abaixo estão relacionadas as proporções de carros e ônibus em algumas cidades brasileiras.
\begin{itemize}
    \item Belo Horizonte: 77% de carros e 23% de ônibus;
    \item Brasília: 91% de carros e 9% de ônibus;
    \item Porto Alegre: 69% de carros e 31% de ônibus;
    \item Recife: 84% de carros e 16% de ônibus;
    \item Rio de Janeiro: 74% de carros e 26% de ônibus;
    \item São Paulo: 88% de carros e 12% de ônibus
    \item Curitiba: 79% de carros e 21% de ônibus.
\end{itemize}

Considerando a situação atual do sistema de transporte público nacional, este projeto tem como escopo a construção uma ferramenta que auxilie o usuário a planejar rotas através do sistema de transporte público.
O objetivo de usar um sistema automatizado para o planejamento de rotas é encontrar automaticamente a maneira mais rápida de se alcançar o destino apenas através do sistema de transporte público e percorrendo, se necessário, pequenos trechos à pé.
Os sistemas de transporte público são, em virtualmente todos os casos, muito menos cômodos do que utilizar um automóvel para se locomover em grandes cidades, e os que tem acesso a um automóvel raramente utilizam o transporte público. Entretanto, mesmo para essas pessoas, o transporte público pode oferecer algumas vantagens, como economia significativa em combustível, manutenção de veículo e estacionamento. Ao facilitar o planejamento de rotas, esta opção de transporte pode se tornar muito mais cômoda e confortável, já que essa tarefa pode não ser trivial.

\section{Motivação}

Muitas vezes, os meios de transporte público não são utilizados puramente devido ao desconhecimento das linhas de transporte presentes em um determinado centro urbano por parte dos usuários. Com este trabalho, espera-se facilitar, do ponto de vista do usuário, o planejamento de trajetos entre localidades situadas nos grandes centros urbanos, com o uso de meios de transporte públicos.

\section{Objetivos}

O presente trabalho tem como objetivos:

\begin{itemize}
	\item Desenvolver uma ferramenta, na forma de serviço web, capaz de auxiliar o planejamento de trajetos urbanos com o uso de transporte público;
	\item Desenvolver uma interface na forma de sítio eletrônico para os usuários do serviço;
	\item Distribuir as tecnologias desenvolvidas sob a forma de software livre, possibilitando que futuramente outras pessoas façam contribuições ao produto desenvolvido;
\end{itemize}



%---------- Segundo Capítulo: Fundamentação --------------

\chapter{Fundamentação}

\section{O Problema do Menor Caminho Multi-Modal}

% citar esse artigo: http://www.scialert.net/fulltext/?doi=jas.2009.3804.3812&org=11#290480_ja

O problema do menor caminho multi-modal consiste em, dada uma rede multi-modal e dois nós nessa rede, a origem e o destino, encontrar o caminho entre a origem e o destino que totaliza o menor custo.
Em uma rede de transporte multi-modal, os nós correspondem às paradas do sistema de transporte, ou aos pontos de origem de destino do trajeto. As arestas correspondem aos deslocamentos entre os nós da rede, e tem associado um custo em tempo e um modo. Os modem correspondem aos serviços ou meios de deslocamento na rede, como por exemplo caminhada ou uma determinada linha de ônibus.
Os serviços de transporte operam segundo tabelas de horários, e portanto para resolver o problema também é necessária a informação do horário de partida no nó de origem.
Uma rede multi-modal pode ser vista como a superposição de diversas redes mono-modais, com a adição de arestas entre elas para representar o tempo de espera associado à troca de serviços.

% vou escrever mais do multi-objetivo depois
Esse problema pode ser generalizado para o caso onde o interessa não é apenas minimizar o tempo de chegada ao destino, mas também minimizar outros objetivos como o custo total do transporte e o número de trocas de serviços. Esta generalização é o problema do menor caminho multi-objetivo.

\section{NoSQL}
%TODO: citar wikipedia =p

NoSQL, ou ``not only SQL'' (não apenas SQL), é uma classe que agrega vários \sigla{SGBD}{Sistema Gerenciador de Banco de Dados}s com características que os diferem dos SGBDs relacionais comuns.
Esses sistemas usualmente permitem que dados sejam armazenados de forma menos rígida em comparação a um esquema de tabelas do modelo relacional, e também nem sempre procuram garantir as propriedades ACID (Atomicidade, Consistência, Isolamento e Durabilidade) comuns nas transações de bancos de dados.
O objetivo desses sistemas é proporcionar um melhor desempenho em algumas aplicações específicas, por exemplo em situações onde existe um grande volume de dados ou de requisições. Esse ganho de desempenho decorre da boa escalabilidade horizontal desses sistemas, ou do fato de eles evitarem junções (produtos cartesianos) entre tabelas.

Existem várias categorias de sistemas NoSQL, de acordo com a implementação e o modo como os dados são armazenados, como por exemplo bancos de dados de documentos, bancos de dados orientados a objetos ou bancos de dados de grafos.
Destes, apenas os bancos de dados de grafos faz parte do escopo do trabalho.

\subsection{Bancos de Dados de Grafos}

%TODO: referências(wiki)

Bancos de dados de grafos são bancos de dados que armazenam os dados na forma de um grafo. Existem diversos modelos de grafos que podem ser utilizados para este propósito, um exemplo de modelo que bastante utilizado é o dos grafos de propriedades, ou seja, grafos que possuem vértices, arestas e que tanto os vértices quanto as arestas podem possuir propriedades.
Assim, ao modelar os dados de um sistema em um banco de dados de grafos, os nós representam as entidades do sistema, enquanto as arestas representam as relações entre essas entidades, e as propriedades contém as demais informações sobre essas entidades e suas relações.

%inserir figura e um exemplo

Esses bancos de dados em geral possuem uma melhor escalabilidade do que os bancos de dados relacionais, já que não utilizam operações de junção nas consultas.
No entanto, sua maior vantagem é que, devido à sua estrutura, são capazes de realizar consultas próprias de grafos de maneira muito mais eficiente, como por exemplo encontrar o menor caminho entre dois nós, ou ainda descobrir se um dado nó é acessível à partir de outro, percorrendo o grafo apenas através de arestas que possuam uma determinada propriedade.

%comentar sobre os bancos existentes, livres ou não, etc?

\section{Métodos Ágeis}

\section{Desenvolvimento Orientado a Testes}

\subsection{Testes de Software}

\subsection{Complexidade Ciclomática}

\section{Arquitetura Cliente-Servidor}

\section{Sistemas Georreferenciados}

\section{Licenciamento de Software}

Licenças de software são instrumentos legais que regulam como um determinado software pode ser distribuído e utilizado.
Softwares, usualmente, não são vendidos, apenas tem seu uso licenciado, então a menos que o software seja colocado em domínio público, os direitos do usuário que licenciou o software e os do dono dos direitos autorais precisam ser bem definidos. Portanto, para que seus direitos autorais sejam respeitados, todo desenvolvedor de software deve se preocupar com o licenciamento de seu trabalho.

Existem dois tipos principais de licenças de software, as proprietárias e as livres e de código aberto. As licenças proprietárias se concentram em garantir que a distribuição do software seja potencialmente rentável, assim como proteger os direitos dos distribuidores. Já as licenças livres e de código aberto se concentram em proteger os direitos dos desenvolvedores ou dos usuários.

\subsection{Software Proprietário} 

Software proprietário é aquele cuja licença restringe ou proíbe diversos usos do produto, como a cópia, redistribuição, engenharia reversa ou modificação. Além disso, o usuário que licencia o software normalmente não tem acesso ao código-fonte e a licença pode ter uma duração limitada.

Quando o usuário adquire uma cópia do software, esta costuma vir acompanhada da licença proprietária, na forma dos termos de uso do software. O software só pode ser utilizado se o usuário concordar com os termos de uso, e pode utilizá-lo apenas da forma prevista nos termos de uso.

\subsection{Software Livre e de Código Aberto}



\subsection{Soluções Proprietárias}

\subsection{Soluções Livres}

\section{Considerações}




%----------- Capítulo 3: Especificação do Software -------------

\chapter{Especificação do Software}
O software em questão consiste em um serviço web que tem como principal funcionalidade responder a rota entre dois pontos informados com menor tempo de viagem utilizando o sistema de transporte público.

\section{Requisitos do Sistema}
A seguir estão listadas os requisitos do sistema.

\subsection{Requisitos Funcionais}
\begin{itemize}
	\item O software deverá retornar para o usuário a rota com menor tempo de viagem entre dois pontos informados.
	\item O software deverá mostrar a rota resultante desenhada em um mapa, diferenciando as linhas de transporte por cor, e também no formato texto como uma sequência de passos.
	\item O software deverá receber do usuário os pontos de partida/destino no formato de endereço e/ou marcando-os no mapa.
	\item O software deverá receber o horário de partida no formato HH:MM.
	\item O software deverá funcionar independente de qual fonte de dados for escolhida, contudo que esta respeite o padrão GTFS.
\end{itemize}

\subsection{Requisitos Não-Funcionais}
\begin{itemize}
	\item O software deverá ser disponibilizado como livre, possibilitando futuras contribuições.
	\item O software deverá ser disponibilizado no formato de um serviço web.
	\item O software devera utilizar o padrão GTFS para o arquivo fonte de dados contendo as rotas do sistema público de transporte de determinada cidade.
	\item O software deverá ser desenvolvido em linguagem Java e, para o núcleo web, javascript.
	\item O software deverá utilizar um banco de dados de grafos para armazenamento das rotas do sistema de transporte público.
\end{itemize}

\section{Arquitetura do Sistema}

\subsection{Core}

\subsection{GTFS Importer}

\subsection{Web Service}

\subsection{Cliente Web}

\section{Considerações}



%----------- Capítulo 4: Metodologia --------------

\chapter{Metodologia}

\section{Estratégia de Desenvolvimento do Software}

\subsection{Desenvolvimento Orientado à Testes}

\subsubsection{Cobertura dos Testes}

\subsection{Revisão de Código}

\section{Ferramentas Utilizadas}

\subsection{Java}

\subsection{Banco de Dados}

% TODO: reforçar o argumento pra usar NoSQL, citar as opções existentes,
% referenciar a tabela, falar pq FlockDB e alguns outros foram desconsiderados

% falar que todos os bancos da tabela são em Java e, portanto, rodam em todas
% as plataformas para as quais há implementação da JVM

% falar que InfoGrid e Neo4j na verdade são dual-license e também oferecem
% uma licença comercial, permitindo que estes sejam utilizados em softwares
% comerciais

\begin{table}[!htb]
	\centering
	\caption{Tabela comparativa entre as opções de Bancos de Dados analisadas}
	\label{tab:bancos}
	\begin{tabular}{lcccc}
		\hline
		& \textbf{HyperGraphDB} & \textbf{InfoGrid} & \textbf{Neo4j} & \textbf{OrientDB} \\
		\hline
		\textbf{Licença} & LGPL & AGLPv3 & AGPLv3 & Apache \\
		\textbf{Lançado em} & ? & ? & 2007 & ? \\
		\textbf{Versão estável} & ? & ? & ? & ? \\
		\textbf{Bindings Java} & Sim & Sim & Sim & Sim \\
		\textbf{Bindings Python} & Não & Não & Sim & Parcial \\
		\textbf{Bindings C/C++} & Não & Não & Não & Não \\
		\textbf{Stand-alone} & Sim & ? & Sim & Sim \\
		\textbf{Embarcado} & Não & ? & Sim & Sim \\
		\textbf{Suporte Blueprints} & Não & Não & Sim & Sim \\
		\hline
	\end{tabular}
	\fonte{Autoria pr\'opria.}
\end{table}

% TODO: argumentar a opção do Neo4j

\subsection{Maven}

\subsection{git}

\subsection{JUnit}

\subsection{Cobertura}

\section{Considerações}



%---------- Quinto Capítulo: Desenvolvimento do Software ----------
\chapter{Desenvolvimento do Software}
\label{chap:desenv}

Utilizando a metodologia e o projeto do sistema apresentados nos capítulos \ref{metod} e \ref{specs}, respectivamente, inicou-se o desenvolvimento efetivo do sistema.
Primeiramente foram configurados o ambiente de desenvolvimento Netbeans juntamente com a ferramenta de gerenciamento de projetos e dependências Maven.
Feito isso, deu-se início a implementação dos módulos \emph{Core}, \emph{Importer}, \emph{Web Service} e Cliente.

%wrappers Neo4j

Nas subseções a seguir serão descritos detalhes a respeito da implementação de cada módulo.

\section{Configuração do ambiente de desenvolvimento}

\section{Core}
O \emph{Core} foi organizado de tal forma que todas as suas entidades com dados originários dos arquivos GTFS possuam suas respectivas \emph{factories}.
Desta forma, toda criação de entidades é realizada através de uma \emph{factory}, centralizando este processo a somente uma classe por entidade.

\subsection{Location}

\subsection{Stop}

\subsection{Route}

\subsection{Trip}

\subsection{StopTime}

\subsection{Connection}

\subsection{DatabaseController}


\section{Importer}

\section{Web Service}

\section{Cliente}

\section{Considerações}

%---------- Sexto Capítulo: Resultados ----------
\chapter{Resultados}

% TODO: resultados

\begin{figure}[!htb]
	\centering
	\includegraphics[width=0.7\textwidth]{./plots/lines_of_code.png}
	\caption[Evolução do número de linhas de código do projeto]{Evolução do número de linhas de código do projeto ao longo do processo de desenvolvimento}
	\fonte{Autoria Própria}
	\label{fig:linesofcode}
\end{figure}

\begin{figure}[!htb]
	\centering
	\includegraphics[width=0.7\textwidth]{./plots/lines_of_code_by_author.png}
	\caption[Evolução do número de linhas de código por programador]{Evolução do número de linhas de código do projeto por programador ao longo do processo de desenvolvimento}
	\fonte{Autoria Própria}
	\label{fig:linesofcodebyauthor}
\end{figure}

\begin{figure}[!htb]
	\centering
	\includegraphics[width=0.7\textwidth]{./plots/day_of_week.png}
	\caption[Número de \emph{commits} por dia da semana]{Número de \emph{commits} por dia da semana}
	\fonte{Autoria Própria}
	\label{fig:dayofweek}
\end{figure}

\begin{figure}[!htb]
	\centering
	\includegraphics[width=0.7\textwidth]{./plots/hour_of_day.png}
	\caption[Número de \emph{commits} por horário]{Número de \emph{commits} por horário}
	\fonte{Autoria Própria}
	\label{fig:hourofday}
\end{figure}


\chapter{Conclusão}
% em torno de 3 paginas
% TODO: Discutir cada um dos objetivos (se foi cumprido e tal)
% TODO: Discutir a metodologia utilizada (se foi bom e tal)
% TODO: Cada uma das considerações dos capítulos deverão entrar na conclusão
% TODO: Discutir a engenharia, relacionar o trabalho com as disciplinas cursadas, estágios, trabalhos
% TODO: Propostas futuras



%---------- Referencias ----------
\bibliography{reflatex} % geracao automatica das referencias a partir do arquivo reflatex.bib


%---------- Apendices (opcionais) ----------
\apendice
\apendice
\chapter{Guia de Desenvolvimento} % FIXME: titulo podre

Este apêndice descreve como obter o código-fonte do projeto e utilizar o \emph{software}.

% TODO

\apendice
\chapter{Exemplo de Uso do \emph{Service}}\label{ape:exemplodeuso}

Este apêndice procura fornecer um claro exemplo de como executar consultas através do componente \emph{Service} do projeto dentro de uma aplicação.
O código apresentado a seguir, em linguagem Python, executa uma consulta no serviço, decodifica o JSON do objeto resposta e, por fim, imprime o resultado na saída padrão.

Vale lembrar que, para executar este exemplo, é necessário ter instalado o interpretador da linguagem Python na versão 2.6 ou superior.

% TODO: falar que o endereço de acesso do service depende do servidor
% falar também que a URL do RouteServlet depende do especificado no arquivo web.xml do onibuscerto-service/src/main/webapp/WEB-INF

% TODO: será que esse exemplo de cliente em Python fica nessa seção mesmo?
% talvez seja melhor colocar me um anexo ou coisa parecida
% outra coisa, não sei como referenciar ele no texto, então ficou assim msm
\lstinputlisting[language=Python]{code/cliente.py}



% ---------- Anexos (opcionais) ----------
%\anexo
%\chapter{Nome do Anexo}

%Use o comando {\ttfamily \textbackslash anexo} e depois comandos {\ttfamily \textbackslash chapter\{\}}
%para gerar t\'itulos de anexos.


% --------- Lista de siglas --------
%\textbf{* Observa\c{c}\~oes:} a lista de siglas nao realiza a ordenacao das siglas em ordem alfabetica
% Em breve isso sera implementado, enquanto isso:
%\textbf{Sugest\~ao:} crie outro arquivo .tex para siglas e utilize o comando \sigla{sigla}{descri\c{c}\~ao}.
%Para incluir este arquivo no final do arquivo, utilize o comando \input{arquivo.tex}.
%Assim, Todas as siglas serao geradas na ultima pagina. Entao, devera excluir a ultima pagina da versao final do arquivo
% PDF do seu documento.


%-------- Citacoes ---------
% - Utilize o comando \citeonline{...} para citacoes com o seguinte formato: Autor et al. (2011).
% Este tipo de formato eh utilizado no comeco do paragrafo. P.ex.: \citeonline{autor2011}

% - Utilize o comando \cite{...} para citacoeses no meio ou final do paragrafo. P.ex.: \cite{autor2011}



%-------- Titulos com nomes cientificos (titulo, capitulos e secoes) ----------
% Regra para escrita de nomes cientificos:
% Os nomes devem ser escritos em italico, 
%a primeira letra do primeiro nome deve ser em maiusculo e o restante em minusculo (inclusive a primeira letra do segundo nome).
% VEJA os exemplos abaixo.
% 
% 1) voce nao quer que a secao fique com uppercase (caixa alta) automaticamente:
%\section[nouppercase]{\MakeUppercase{Estudo dos efeitos da radiacao ultravioleta C e TFD em celulas de} {\textit{Saccharomyces boulardii}}
%
% 2) por padrao os cases (maiusculas/minuscula) sao ajustados automaticamente, voce nao precisa usar makeuppercase e afins.
% \section{Introducao} % a introducao sera posta no texto como INTRODUCAO, automaticamente, como a norma indica.


\end{document}
