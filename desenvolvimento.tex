%---------- Quinto Capítulo: Desenvolvimento do Software ----------
\chapter{Desenvolvimento do Software}
\label{chap:desenv}

Utilizando a metodologia e o projeto do sistema apresentados nos capítulos \ref{metod} e \ref{specs}, respectivamente, inicou-se o desenvolvimento efetivo do sistema.
Primeiramente foi criado um projeto Maven principal denominado \texttt{onibuscerto}, o qual foi dividido em quatro subprojetos: 
\begin{itemize}
	\item \texttt{onibuscerto-api}: contém a classe que representa coordenadas geográficas e o objeto resposta contendo informações da rota ao Cliente.
	\item \texttt{onibuscerto-core}: consiste no módulo \emph{Core} definido no projeto do software (seção \ref{specs}). 
	Contém as representações das entidades originárias dos arquivos GTFS e faz interface do sistema com o banco de dados.
	\item \texttt{onibuscerto-importer}: consiste no módulo \emph{Importer} definido no projeto do software (seção \ref{specs}).
	Responsável pela importação dos dados contidos nos arquivos GTFS para o sistema através das funcionalidades do \emph{Core}.
	\item \texttt{onibuscerto-service}: consiste nos módulos \emph{Web Service} e Cliente definidos na seção \ref{specs}. 
	Pretende-se futuramente separar este subprojeto em dois outros distintos representando tais módulos.
\end{itemize}

Nas subseções a seguir serão descritos detalhes a respeito da implementação de cada subprojeto.

\section{Configuração do ambiente de desenvolvimento}

\section{onibuscerto-core}
O \emph{Core} foi organizado de tal forma que todas as suas entidades com dados originários dos arquivos GTFS possuam suas respectivas \emph{factories}.
Desta forma, toda criação de entidades é realizada através de uma \emph{factory}, centralizando este processo a somente uma classe por entidade.

%wrappers Neo4j


\subsection{Location}

\subsection{Stop}

\subsection{Route}

\subsection{Trip}

\subsection{StopTime}

\subsection{Connection}

\subsection{DatabaseController}


\section{onibuscerto-importer}

\section{onibuscerto-service}

O subprojeto \texttt{onibuscerto-service} é onde encontram-se os Servlets responsáveis por executar consultas na base de dados construída pelo \emph{Importer}.
Os Servlets funcionam sob a forma de \emph{web services} e, sendo assim, todas as consultas são requisitadas pelos clientes através de requisições \sigla{HTTP}{Hypertext Transfer Protocol} do tipo GET ou POST.
As implementações dos Servlets residem no pacote \texttt{com.onibuscerto.service.servlets}.

Atualmente, apenas consultas que determinam a rota com o menor tempo de viagem entre duas coordenadas geográficas estão implementadas.
O Servlet responsável por executar esse tipo de consulta está implementado na classe \texttt{RouteServlet}.
No início do seu ciclo de vida, este Servlet é responsável por obter uma instância da classe \texttt{DatabaseController} do \emph{Core}, a qual será utilizada para acessar o banco de dados e resolver as consultas.
Em seguida, assim como um Servlet comum, este muda para um estado no qual está disponível para atender requisições dos clientes.
Por fim, ao ser destruído, o \texttt{RouteServlet} deve chamar o método \texttt{close} do \texttt{DatabaseController}, com o objetivo de fechar a conexão com o banco de dados.
O ciclo de vida completo do Servlet é ilustrado na Figura \ref{fig:servletciclo}.

\begin{figure}[!htb]
	\centering
	\includegraphics[width=0.4\textwidth]{./servletciclo.png}
	\caption[Ciclo de vida de um Servlet]{Ciclo de vida de um Servlet}
	\fonte{\citeonline{infocenter}}
	\label{fig:servletciclo}
\end{figure}

\section{Considerações}
