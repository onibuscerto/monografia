%----------- Capítulo 4: Metodologia --------------

\chapter{Metodologia}

\section{Estratégia de Desenvolvimento do Software}

\subsection{Desenvolvimento Orientado à Testes}

\subsubsection{Cobertura dos Testes}

\subsection{Revisão de Código}

\section{Ferramentas Utilizadas}

\subsection{Java}

\subsection{Banco de Dados}

% TODO: reforçar o argumento pra usar NoSQL, citar as opções existentes,
% referenciar a tabela, falar pq FlockDB e alguns outros foram desconsiderados

% falar que todos os bancos da tabela são em Java e, portanto, rodam em todas
% as plataformas para as quais há implementação da JVM

% falar que InfoGrid e Neo4j na verdade são dual-license e também oferecem
% uma licença comercial, permitindo que estes sejam utilizados em softwares
% comerciais

\begin{table}[!htb]
	\centering
	\caption{Tabela comparativa entre as opções de Bancos de Dados analisadas}
	\label{tab:bancos}
	\begin{tabular}{lcccc}
		\hline
		& \textbf{HyperGraphDB} & \textbf{InfoGrid} & \textbf{Neo4j} & \textbf{OrientDB} \\
		\hline
		\textbf{Licença} & LGPL & AGLPv3 & AGPLv3 & Apache \\
		\textbf{Lançado em} & ? & ? & 2007 & ? \\
		\textbf{Versão estável} & ? & ? & ? & ? \\
		\textbf{Bindings Java} & Sim & Sim & Sim & Sim \\
		\textbf{Bindings Python} & Não & Não & Sim & Parcial \\
		\textbf{Bindings C/C++} & Não & Não & Não & Não \\
		\textbf{Stand-alone} & Sim & ? & Sim & Sim \\
		\textbf{Embarcado} & Não & ? & Sim & Sim \\
		\textbf{Suporte Blueprints} & Não & Não & Sim & Sim \\
		\hline
	\end{tabular}
	\fonte{Autoria pr\'opria.}
\end{table}

% TODO: argumentar a opção do Neo4j

\subsection{Maven}

\subsection{git}

\subsection{JUnit}

\subsection{Cobertura}

\section{Considerações}

