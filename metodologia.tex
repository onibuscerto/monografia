%----------- Capítulo 4: Metodologia --------------

\chapter{Metodologia}

\section{Estratégia de Desenvolvimento do Software}

\subsection{Desenvolvimento Orientado à Testes}

\subsubsection{Cobertura dos Testes}

\subsection{Revisão de Código}

\section{Ferramentas Utilizadas}

\subsection{Java}

\subsection{Banco de Dados}

% TODO: reforçar o argumento pra usar NoSQL, citar as opções existentes,
% referenciar a tabela, falar pq FlockDB e alguns outros foram desconsiderados

% falar que todos os bancos da tabela são em Java e, portanto, rodam em todas
% as plataformas para as quais há implementação da JVM

% falar que InfoGrid e Neo4j na verdade são dual-license e também oferecem
% uma licença comercial, permitindo que estes sejam utilizados em softwares
% comerciais

Como citado anteriormente, os bancos de dados NoSQL permitem que os dados sejam armazenados de uma forma menos rígida que nos bancos de dados relacionais.
Por este motivo, os bancos de dados NoSQL apresentam uma grande vantagem com relação aos demais para armazenar e extrair informações rapidamente de bases de dados que possam ser modeladas como grafos. % TODO: citar algo

Dada a natureza do problema sendo estudado, naturalmente a equipe optou por utilizar um dos bancos de dados NoSQL que enfatizam em grafos.
Entre as principais opções disponíveis que foram consideradas para este projeto estão o HyperGraphDB, desenvolvido pela Kobrix Software Inc., o InfoGrid, desenvolvido pela NetMesh Inc., o Neo4j, desenvolvido pela Neo Technology Inc. e o OrientDB, desenvolvido pela Orient Technologies.
Estes bancos de dados são apresentados na Tabela \ref{tab:bancos}.

Outros bancos de dados disponíveis na Internet também foram inicialmente analisados, como o FlockDB, desenvolvido pela Twitter Inc..
No entanto, estes foram desconsiderados posteriormente por não possuírem documentação suficiente no momento em que foi dado início ao desenvolvimento do projeto ou focarem demais em um problema específico, como é o caso do próprio FlockDB, que é utilizado para armazenar as relações sociais entre os usuários do serviço do Twitter: ``quem segue quem'' e ``quem é seguido por quem''.

\begin{table}[!htb]
	\centering
	\caption{Tabela comparativa entre as opções de Bancos de Dados analisadas}
	\label{tab:bancos}
	\begin{tabular}{lcccc}
		\hline
		& \textbf{HyperGraphDB} & \textbf{InfoGrid} & \textbf{Neo4j} & \textbf{OrientDB} \\
		\hline
		\textbf{Licença} & LGPL & AGLPv3 & AGPLv3 & Apache \\
		\textbf{Iniciado em} & 2005 & ? & 2003 & ? \\
		\textbf{Versão estável} & 1.1 & 2.9.5 & 1.4.2 & 0.9.25 \\
		\textbf{Data versão estável} & Dezembro 2010 & Agosto 2011 & Setembro 2011 & Março 2011 \\
		\textbf{Bindings Java} & Sim & Sim & Sim & Sim \\
		\textbf{Bindings Python} & Não & Não & Sim & Parcial \\
		\textbf{Bindings C/C++} & Não & Não & Não & Não \\
		\textbf{Stand-alone} & Sim & ? & Sim & Sim \\
		\textbf{Embarcado} & Não & ? & Sim & Sim \\
		\textbf{Suporte Blueprints} & Não & Não & Sim & Sim \\
		\hline
	\end{tabular}
	\fonte{Autoria pr\'opria.}
\end{table}

Como é possível ver pela Tabela \ref{tab:bancos}

% TODO: argumentar a opção do Neo4j

\subsection{Maven}

\subsection{git}

\subsection{JUnit}

\subsection{Cobertura}

\section{Considerações}

